\documentclass[10pt]{beamer}
\usetheme{metropolis}
% all imports
\usepackage[utf8]{inputenc}
\usepackage[T1]{fontenc}
\usepackage{lmodern}
\usepackage{appendixnumberbeamer}
\usepackage{hyperref}
\usepackage{booktabs}
\usepackage{amsmath}
\usepackage{bm}
\usepackage[scale=2]{ccicons}
\usepackage[outputdir=build]{minted}
\usepackage{pgfplots}
\usepackage{array,colortbl,xcolor}
\usepgfplotslibrary{dateplot}
\usepackage{setspace}
\usepackage{etoolbox}
\usepackage{xspace}
\usepackage{tikz}
\usetikzlibrary{shapes,arrows,positioning,fit,backgrounds}
\usepackage{tkz-euclide}
\usepackage{soul}
\usepackage{ragged2e}
\usepackage{algorithm,algorithmic}


\AtBeginEnvironment{quote}{\singlespacing}

% new commands
\newcommand{\themename}{\textbf{\textsc{metropolis}}\xspace}
\newcommand{\vect}[1]{\bm{#1}}
\newcommand{\myprime}[1]{{#1}^{\prime}}
\newcommand{\grad}[2]{\nabla_{#1} {#2}}
\newcommand{\dotp}[2]{{#1}^{\top}{#2}}
\newcommand{\dotpPright}[2]{{#1}^{\top}\left({#2}\right)}
\newcommand{\outerp}[2]{\left({#1}\right){#2}^{\top}}
\newcommand{\Jacobian}[2]{\frac{\partial #1}{\partial #2}}
\newcommand{\Vocab}{\mathbb{V}}
\DeclareMathOperator*{\argmin}{arg\,min}
\DeclareMathOperator*{\argmax}{arg\,max}
\DeclareMathOperator{\E}{\mathbb{E}}


% Quote with author reference at the end
\let\oldquote\quote
\let\endoldquote\endquote
\renewenvironment{quote}[2][]
  {\if\relax\detokenize{#1}\relax
     \def\quoteauthor{#2}%
   \else
     \def\quoteauthor{#2~---~#1}%
   \fi
   \oldquote}
  {\par\nobreak\smallskip\hfill(\quoteauthor)%
   \endoldquote\addvspace{\bigskipamount}}
%-----------------------------------------   


% definitions
\definecolor{blue}{RGB}{159, 192, 176}
\definecolor{blue2}{RGB}{38,139,210}
\definecolor{green}{RGB}{160, 227, 127}
\definecolor{green2}{RGB}{132, 164, 76}
\definecolor{orange}{RGB}{243, 188, 125}
\definecolor{red}{RGB}{253, 123, 84}
\definecolor{nephritis}{RGB}{39, 174, 96}
\definecolor{emerald}{RGB}{46, 204, 113}
\definecolor{turquoise}{RGB}{39, 174, 96}
\definecolor{green-sea}{RGB}{22, 160, 133}
\definecolor{base02}{RGB}{7,54,66}
\definecolor{base03}{RGB}{0,43,54}
\definecolor{cyan}{RGB}{42,161,152}
% Tikzstyles for Computation Graphs

% nodes
\tikzstyle{noop} = [circle, draw=none, fill=red, minimum size = 10pt]
\tikzstyle{op} = [circle, draw=red, line width=1.5pt, fill=red!70, text=black, text centered, font=\bf \normalsize, minimum size = 25pt]
\tikzstyle{op2} = [circle, draw=orange, line width=1.5pt, fill=orange!70, text=black, text centered, font=\bf \normalsize, minimum size = 25pt]
\tikzstyle{op3} = [circle, draw=orange, line width=1.5pt, fill=orange!70, text=black, text centered, font=\bf \scriptsize, minimum size = 7pt]
\tikzstyle{placeholder} = [circle, draw=red, line width=1.5pt, fill=red!30, text=black, text centered, font=\bf  \normalsize, minimum size = 25pt]
\tikzstyle{state} = [circle, draw=blue, line width=1.5pt, fill=blue!70, text=black, text centered, font=\bf \normalsize, minimum size = 25pt]
\tikzstyle{gradient} = [circle, draw=nephritis, line width=1.5pt, fill=nephritis!60, text=black, text centered, font=\bf \normalsize, minimum size = 25pt]
\tikzstyle{gradient2} = [circle, draw=green2, line width=1.5pt, fill=green2!60, text=black, text centered, font=\bf \normalsize, minimum size = 25pt]
\tikzstyle{textonly} = [draw=none, fill=none, text centered, font=\bf \normalsize]

% edges
% \tikzstyle{tedge}  = [draw, thick, >=stealth, ->]
\tikzstyle{tedge}  = [draw, thick, >=latex, ->]
\tikzstyle{tedge_dashed}  = [draw, thick, >=latex, ->, dashed]

% namedscope
\tikzstyle{namedscope} = [circle, draw=orange, line width=1.5pt, fill=orange!60, align=center, inner sep=0pt]

% \tikzstyle{container} = [draw=none, rectangle, dotted, inner ysep=1.5em]
% \tikzstyle{novertex} = [draw=none, fill=none, text centered]
% \tikzstyle{predicate} = [ellipse, draw, thick, text centered, rounded corners, minimum size=30pt]
% \tikzstyle{aux} = [rectangle, draw, thick, text centered, rounded corners, minimum size=30pt]
% \tikzstyle{ledge}  = [draw, dashed, thick, >=stealth, ->]
% \tikzstyle{pedge}  = [draw, thick, >=stealth, ->]


\title{MAC0460 - Introdução ao aprendizado de máquina \\ \vspace{0.2 cm} Back-propagation 1} 
\date{\today}

\author{
  Felipe Salvatore\\
  \url{https://felipessalvatore.github.io/}\vspace{0.4 cm}
  \and\\ 
  Nina S. T. Hirata\\
  \url{https://www.ime.usp.br/~nina/}
  \vspace{0.4 cm}
}

\institute{\textbf{IME-USP}: Institute of Mathematics and Statistics, University of São Paulo}




\begin{document}
\nocite{DeepLearningbook}
\nocite{VectorCalculus}
\nocite{graphsbackprop}
\nocite{DBLP:journals/corr/abs-1802-01528}

\maketitle

\begin{frame}{Definição de Jacobiano}
\Large{
\begin{equation*}
f(x, y) = x + y
\end{equation*}

\vspace{0.5 cm}

\begin{equation*}
f\left( \vect{u} = \begin{bmatrix}x \\ y\end{bmatrix}\right) = x + y
\end{equation*}


\vspace{0.5 cm}

\begin{equation*}
\grad{\vect{u}}{f} = \Jacobian{f}{\vect{u}} = \begin{bmatrix} \Jacobian{f}{x} & \Jacobian{f}{y} \end{bmatrix}
\end{equation*}
}
\end{frame}

\begin{frame}{Convenção de shape}
\large{
\begin{figure}[H]
\begin{center}
\begin{tabular}{|c|c|c|}
\hline
\cellcolor{orange!100}  & \cellcolor{orange!100} $shape(\vect{x}) = 1\times1$ & \cellcolor{orange!100} $shape(\vect{x}) = n\times1$ \\ \hline
 \cellcolor{orange!100} $shape(\vect{f}) = 1\times1$ & $shape(\Jacobian{\vect{f}}{\vect{x}}) = 1\times1$ & $shape(\Jacobian{\vect{f}}{\vect{x}}) = 1\times n$ \\ \hline
\cellcolor{orange!100} $shape(\vect{f}) = m\times1$ & $shape(\Jacobian{\vect{f}}{\vect{x}}) = m\times1$ & $shape(\Jacobian{\vect{f}}{\vect{x}}) = m\times n$ \\ \hline
\end{tabular}
\end{center}
\end{figure}
}
\end{frame}


\begin{frame}{Discussão}
\large{
Dado $\vect{u} \in \mathbb{R}^{n}$ e $f:\mathbb{R}^{n} \rightarrow \mathbb{R}$ podemos definir $\grad{\vect{u}}{f} = \Jacobian{f}{\vect{u}}$ como um vetor coluna. 

\vspace{0.3cm}
\begin{itemize}
\item (\alert{positivo}) $u  + \grad{\vect{u}}{f}$ faz sentido.

\item[]


\item (\alert{negativo}) quando $\vect{x} \in \mathbb{R}^{n}$, $f:\mathbb{R}^{k} \rightarrow \mathbb{R}$, $\vect{g}:\mathbb{R}^{n} \rightarrow \mathbb{R}^{k}$, $\vect{y} = \vect{g}(\vect{x})$ e $z = f(\vect{y})$, a regra da cadeia tem um formato menos intuitivo.

\begin{equation*}
\grad{\vect{x}}{z} = (\Jacobian{\vect{y}}{\vect{x}})^{\top} \grad{\vect{y}}{z}
\end{equation*}

\end{itemize}
}
\end{frame}



\begin{frame}{Operações básicas: produto vetor-escalar}
\Large{
$\vect{x}\in \mathbb{R}^{n}$ e $\alpha \in \mathbb{R}$

\vspace{0.3 cm}

\begin{equation*}
\vect{u} = \vect{x}\alpha
\end{equation*}

\vspace{0.3 cm}
\begin{itemize}
\item $\Jacobian{\vect{u}}{\vect{x}} = \underbrace{diag(\vect{1}\alpha)}_{\tiny{n \times n}}$
\item[]
\item $\Jacobian{\vect{u}}{\alpha} = \underbrace{\vect{x}}_{\tiny{n \times 1}}$
\end{itemize}
}
\end{frame}

\begin{frame}{Operações básicas: soma}
\Large{
$\vect{x},\vect{y} \in \mathbb{R}^{n}$

\vspace{0.3 cm}

\begin{equation*}
\vect{u} = \vect{x} + \vect{y}
\end{equation*}

\vspace{0.3 cm}
\begin{itemize}
\item $\Jacobian{\vect{u}}{\vect{x}} = diag(\vect{1}) = \underbrace{\vect{I}}_{\tiny{n \times n}}$
\item[]
\item $\Jacobian{\vect{u}}{\vect{y}} = diag(\vect{1}) = \underbrace{\vect{I}}_{\tiny{n \times n}}$
\end{itemize}
}
\end{frame}

\begin{frame}{Operações básicas: Hadamard product}
\Large{
$\vect{x},\vect{y} \in \mathbb{R}^{n}$

\vspace{0.3 cm}

\begin{equation*}
\vect{u} = \vect{x} \odot \vect{y}
\end{equation*}


\vspace{0.3 cm}
\begin{itemize}
\item $\Jacobian{\vect{u}}{\vect{x}} = \underbrace{diag(\vect{y})}_{\tiny{n \times n}}$
\item[]
\item $\Jacobian{\vect{u}}{\vect{y}} = \underbrace{diag(\vect{x})}_{\tiny{n \times n}}$
\end{itemize}
}
\end{frame}

\begin{frame}{Operações básicas: função escalar aplicada em vetor}
\Large{


$\vect{x}\in \mathbb{R}^{n}$ e $h:\mathbb{R} \rightarrow \mathbb{R}$ é uma função diferenciável.

\vspace{0.1 cm}

\begin{equation*}
\vect{u} = h(\vect{x}) = \begin{bmatrix}h(x_1)\\ h(x_2)\\ \vdots \\h(x_n)\end{bmatrix}
\end{equation*}

\vspace{0.1 cm}
\begin{itemize}
\item $\Jacobian{\vect{u}}{\vect{x}} = \underbrace{diag(h^{\prime}(\vect{x}))}_{\tiny{n \times n}} \;\;\;\;\;\;\; \text{onde} \;\; h^{\prime}(\vect{x}) = \begin{bmatrix}\frac{d h(x_1)}{d x_1}\\ \frac{d h(x_2)}{d x_2}\\ \vdots \\\frac{d h(x_n)}{d x_n}\end{bmatrix}$
\end{itemize}
}
\end{frame}


\begin{frame}{Operações básicas: redução por soma}
\Large{
$\vect{x}\in \mathbb{R}^{n}$

\vspace{0.3 cm}

\begin{equation*}
u = sum(\vect{x}) = \sum_{i=1}^{n} x_{i}
\end{equation*}

\vspace{0.3 cm}
\begin{itemize}
\item $\Jacobian{u}{\vect{x}} = \underbrace{\vect{1}^{\top}}_{\tiny{1 \times n}}$
\end{itemize}
}
\end{frame}


\begin{frame}{Operações básicas: multiplicação matriz-vetor}
\Large{
$\vect{x}\in \mathbb{R}^{n}$, $\vect{W} \in \mathbb{R}^{d\times n}$ 

\vspace{0.3 cm}

\begin{equation*}
\vect{u} = \vect{W}\vect{x}
\end{equation*}



\vspace{0.3 cm}
\begin{itemize}
\item $\Jacobian{\vect{u}}{\vect{x}} = \underbrace{\vect{W}}_{d \times n}$
\item[]



\item $\underbrace{\Jacobian{\vect{u}}{\vect{W}}}_{d\times d \times n}$ tal que $\Jacobian{\vect{u}}{\vect{W}} _{i,j,k} = \begin{cases}
0, \text{ se } i \neq j\\
x_k, \text{ se } i = j\\
\end{cases}$
\end{itemize}
}
\end{frame}


\begin{frame}{Revisão: regra da cadeia}

Dados $x, u_1(x), \dots, u_n(x) \in \mathbb{R}$ e $f:\mathbb{R}^{n} \rightarrow \mathbb{R}$ temos que cada $u_i$ varia dado uma variação em $x$. Assim a regra da cadeia para várias variáveis é definida como:

\vspace{0.3 cm}

\begin{align*}
\Jacobian{f(u_1, \dots, u_n)}{x} &= \Jacobian{f(u_1, \dots, u_n)}{u_1}\Jacobian{u_1}{x} + \dots +  \Jacobian{f(u_1, \dots, u_n)}{u_n}\Jacobian{u_n}{x}\\
&\\
&= \sum_{i=1}^{n} \Jacobian{f(u_1, \dots, u_n)}{u_i}\Jacobian{u_i}{x}\\
\end{align*}
\end{frame}


\begin{frame}{Regra da cadeia, caso vetorial}
\large{
$\vect{x}\in \mathbb{R}^{n}$, $\vect{f}:\mathbb{R}^{k} \rightarrow \mathbb{R}^{m}$ e $\vect{g}:\mathbb{R}^{n} \rightarrow \mathbb{R}^{k}$.

\vspace{1.0 cm}

$\vect{g}(\vect{x}) = \begin{bmatrix}g_1(x_1, \dots, x_n)\\ g_2(x_1, \dots, x_n)\\ \vdots \\g_k(x_1, \dots, x_n)\end{bmatrix} \;\;\;\;\;\;\;\; \vect{f}(\vect{g}(\vect{x})) = \begin{bmatrix}f_1(g_1, \dots, g_k)\\ f_2(g_1, \dots, g_k)\\ \vdots \\f_m(g_1, \dots, g_k)\end{bmatrix}$
}
\end{frame}


\begin{frame}{Regra da cadeia, caso vetorial}
\large{
\begin{align*}
\Jacobian{\vect{f}(\vect{g}(\vect{x}))}{\vect{x}}_{i,j} &= \Jacobian{f_i}{x_j}\\
&\\
&= \Jacobian{f_i(g_1, \dots, g_k)}{x_j}\\
&\\
&= \sum_{s=1}^{k} \Jacobian{f_i(g_1, \dots, g_k)}{g_s}\Jacobian{g_s}{x_j}\\
&\\
&= \Jacobian{\vect{f}}{\vect{g}} \Jacobian{\vect{g}}{\vect{x}}_{i,j}
\end{align*}
}
\end{frame}


\begin{frame}{Regra da cadeia, caso vetorial}
\Large{

\begin{equation*}
\underbrace{\Jacobian{\vect{f}(\vect{g}(\vect{x}))}{\vect{x}}}_{m \times n}  = \underbrace{\Jacobian{\vect{f}}{\vect{g}}}_{m \times k} \, \underbrace{\Jacobian{\vect{g}}{\vect{x}}}_{k \times n}
\end{equation*}
}
\end{frame}

\begin{frame}{Regra da cadeia, caso vetorial}
\Large{
Dados $\vect{x} \in  \mathbb{R}^{n}$,  $\vect{u}^{(1)}(\vect{x}), \dots, \vect{u}^{(s)}(\vect{x}) \in \mathbb{R}^{k}$ e $\vect{f}:\mathbb{R}^{s \times k} \rightarrow \mathbb{R}^{m}$:

\begin{equation*}
\Jacobian{\vect{f}(\vect{u}^{(1)}, \dots, \vect{u}^{(s)})}{\vect{x}} = \sum_{i=1}^{s} \Jacobian{\vect{f}(\vect{u}^{(1)}, \dots, \vect{u}^{(s)})}{\vect{u}^{(i)}} \Jacobian{\vect{u}^{(i)}}{\vect{x}}
\end{equation*}
}
\end{frame}

\begin{frame}{Novas operações: subtração}

\Large{

$\vect{x},\vect{y} \in \mathbb{R}^{n}$

\vspace{0.3 cm}

\begin{equation*}
\vect{u} = \vect{x} - \vect{y}
\end{equation*}

$\vect{u} = \vect{f}(\vect{g}(\vect{y})) = \vect{x} + \vect{g}(\vect{y})$

$\vect{g}(\vect{y}) = - \vect{y}$


\vspace{0.3 cm}
\begin{itemize}
\item $\Jacobian{\vect{u}}{\vect{x}} = \underbrace{\vect{I}}_{\tiny{n \times n}}$
\item[]
\item $\Jacobian{\vect{u}}{\vect{y}} = \Jacobian{\vect{f}}{\vect{g}} \Jacobian{\vect{g}}{\vect{y}} = \underbrace{\vect{I}}_{\tiny{n \times n}} (\underbrace{-\vect{I}}_{\tiny{n \times n}}) = \underbrace{-\vect{I}}_{\tiny{n \times n}}$
\end{itemize}
}
\end{frame}


\begin{frame}{Novas operações: produto escalar}
\Large{

$\vect{x},\vect{y} \in \mathbb{R}^{n}$

\vspace{0.1 cm}

\begin{equation*}
u = \vect{x}^{\top}\vect{y}
\end{equation*}

\vspace{0.2 cm}
$u = \vect{f}(\vect{g}(\vect{x},\vect{y})) = sum(\vect{g}(\vect{x}, \vect{y}))$

\vspace{0.2 cm}
$\vect{g}(\vect{x}, \vect{y}) = \vect{x} \odot \vect{y}$


\vspace{0.2 cm}
\begin{itemize}
\item $\Jacobian{u}{\vect{x}} = \Jacobian{\vect{f}}{\vect{g}} \Jacobian{\vect{g}}{\vect{x}} = \underbrace{\vect{1}^{\top}}_{\tiny{1 \times n}} \underbrace{diag(\vect{y})}_{\tiny{n \times n}} \;\; = \underbrace{\vect{y}^{\top}}_{\tiny{1 \times n}}$
\item[]
\item $\Jacobian{u}{\vect{y}} = \Jacobian{\vect{f}}{\vect{g}} \Jacobian{\vect{g}}{\vect{y}} = \;\; \underbrace{\vect{1}^{\top}}_{\tiny{1 \times n}} \underbrace{diag(\vect{x})}_{\tiny{n \times n}} \;\; = \underbrace{\vect{x}^{\top}}_{\tiny{1 \times n}}$
\end{itemize}
}
\end{frame}

\begin{frame}{Exemplo 1: regressão linear}
\large{

$\vect{x},\vect{w} \in \mathbb{R}^{n}$ e $y \in \mathbb{R}$
\begin{equation*}
L = f(g(\hat{y}))
\end{equation*}

$\hat{y} = \vect{w}^{\top}\vect{x}$

$g(\hat{y}) = \hat{y} - y$

$f(g(\hat{y})) = g(\hat{y})^2$


\begin{align*}
\Jacobian{L}{\vect{w}} &= \Jacobian{f}{g}\Jacobian{g}{\hat{y}}\Jacobian{\hat{y}}{\vect{w}}\\
% &\\
&= \underbrace{2(\hat{y} - y)}_{\tiny{1 \times 1}} \underbrace{1}_{\tiny{1 \times 1}} \underbrace{\vect{x}^{\top}}_{\tiny{1 \times n}} \\
% &\\
&= \underbrace{2(\hat{y} - y)\vect{x}^{\top}}_{\tiny{1 \times n}} 
\end{align*}
}
\end{frame}

\begin{frame}{Novas operações: transformação afim}
\large{
$\vect{x}\in \mathbb{R}^{n}$, $\vect{W} \in \mathbb{R}^{d\times n}$, $\vect{b}\in \mathbb{R}^{d}$
\vspace{0.1 cm}
\begin{equation*}
\vect{u} = \vect{W}\vect{x} + \vect{b}
\end{equation*}

\vspace{0.2 cm}
$\vect{u} = \vect{f}(\vect{g}(\vect{W},\vect{x})) =\vect{g}(\vect{W}, \vect{x}) + \vect{b}$

\vspace{0.2 cm}
$\vect{g}(\vect{W}, \vect{x}) = \vect{W}\vect{x}$

\vspace{0.2 cm}
\begin{itemize}
\item $\Jacobian{\vect{u}}{\vect{x}} = \Jacobian{\vect{f}}{\vect{g}}\Jacobian{\vect{g}}{\vect{x}} = \underbrace{\vect{I}}_{\tiny{d \times d}}\underbrace{\vect{W}}_{\tiny{d \times n}} = \underbrace{\vect{W}}_{\tiny{d \times n}}$
\item[]
\item $\Jacobian{\vect{u}}{\vect{b}} = \underbrace{\vect{I}}_{\tiny{d \times d}}$
\item[]
\item $\Jacobian{\vect{u}}{\vect{W}} = \Jacobian{\vect{f}}{\vect{g}}\Jacobian{\vect{g}}{\vect{W}} = \underbrace{\vect{I}}_{\tiny{d \times d}}\underbrace{\Jacobian{\vect{g}}{\vect{W}}}_{\tiny{d \times d \times n}} = \underbrace{\Jacobian{\vect{g}}{\vect{W}}}_{\tiny{d \times d \times n}}$
\end{itemize}
}
\end{frame}

\begin{frame}{Novas operações: ativação}
\large{

$\vect{x}\in \mathbb{R}^{n}$, $\vect{W} \in \mathbb{R}^{d\times n}$, $\vect{b}\in \mathbb{R}^{d}$, $h:\mathbb{R} \rightarrow \mathbb{R}$

\vspace{0.1 cm}

\begin{equation*}
\vect{u} = h(\vect{W}\vect{x} + \vect{b})
\end{equation*}

\vspace{0.1 cm}

$\vect{u} = \vect{f}(\vect{g}(\vect{W},\vect{x},\vect{b})) = h(\vect{g}(\vect{W}, \vect{x},\vect{b}))$

\vspace{0.1 cm}

$\vect{g}(\vect{W}, \vect{x}, \vect{b}) = \vect{W}\vect{x} + \vect{b}$

\vspace{0.1 cm}

\begin{align*}
\bullet \; \; \Jacobian{\vect{u}}{\vect{W}} &= \Jacobian{\vect{f}}{\vect{g}}\Jacobian{\vect{g}}{\vect{W}}\\
&\\
&= \underbrace{diag(h^{\prime}(\vect{W}\vect{x} + \vect{b}))}_{\tiny{d \times d}} \underbrace{\Jacobian{\vect{g}}{\vect{W}}}_{\tiny{d \times d \times n}}\\
\end{align*}
}
\end{frame}


\begin{frame}{Novas operações: ativação}
\large{
\begin{equation*}
\underbrace{\Jacobian{\vect{u}}{\vect{W}}}_{\tiny{d \times d \times n}} \text{ tal que }  \Jacobian{\vect{u}}{\vect{W}}_{i,j,k} = \begin{cases}
0, \text{ se } i \neq j\\
h^{\prime}(\vect{W}\vect{x} + \vect{b})_i x_k, \text{ se } i = j\\
\end{cases}
\end{equation*}

\begin{align*}
\bullet \; \; \Jacobian{\vect{u}}{\vect{b}} &= \Jacobian{\vect{f}}{\vect{g}}\Jacobian{\vect{g}}{\vect{b}}\\
&\\
&= \underbrace{diag(h^{\prime}(\vect{W}\vect{x} + \vect{b}))}_{\tiny{d \times d}} \underbrace{\vect{I}}_{\tiny{d \times d}}\\
&\\
&= \underbrace{diag(h^{\prime}(\vect{W}\vect{x} + \vect{b}))}_{\tiny{d \times d}}\\
\end{align*}
}
\end{frame}


\begin{frame}{Novas operações: ativação}
\large{
\vspace{0.5 cm}

\begin{align*}
\bullet \; \; \Jacobian{\vect{u}}{\vect{x}} &= \Jacobian{\vect{f}}{\vect{g}}\Jacobian{\vect{g}}{\vect{x}}\\
&\\
&= \underbrace{diag(h^{\prime}(\vect{W}\vect{x} + \vect{b}))}_{\tiny{d \times d}} \underbrace{\vect{W}}_{\tiny{d \times n}}\\
&\\
&= \begin{bmatrix}h^{\prime}(\vect{W}\vect{x} + \vect{b})_1 w_{1,1}& \dots & h^{\prime}(\vect{W}\vect{x} + \vect{b})_1 w_{1,n}\\ \vdots & \ddots & \vdots\\ h^{\prime}(\vect{W}\vect{x} + \vect{b})_d w_{d,1}& \dots & h^{\prime}(\vect{W}\vect{x} + \vect{b})_d w_{d,n} \end{bmatrix}_{\tiny{d \times n}}\\
\end{align*}
}
\end{frame}


\begin{frame}{Novas operações: softmax}
\large{
$\vect{x}\in \mathbb{R}^{n}$

\begin{equation*}
\vect{u} = \vect{s}(\vect{x})
\end{equation*}


$\vect{f}(\vect{x}) = exp(\vect{x})$

$g(\vect{f}) = sum(\vect{f})$

$h(g) = g^{-1}$

$\vect{t}(\vect{f},h) = \vect{f}h$

$\vect{u} = \vect{s}(\vect{x}) = \vect{t}(\vect{f},h)$

\begin{align*}
\bullet \;\; \Jacobian{\vect{u}}{\vect{x}} &= \Jacobian{\vect{t}}{\vect{f}}\Jacobian{\vect{f}}{\vect{x}}  + \Jacobian{\vect{t}}{h}\Jacobian{h}{\vect{x}}\\
\end{align*}
}
\end{frame}

\begin{frame}{Novas operações: softmax}
\large{
\begin{align*}
\Jacobian{\vect{t}}{\vect{f}}\Jacobian{\vect{f}}{\vect{x}} & = \Jacobian{\vect{t}}{\vect{f}}\Jacobian{\vect{f}}{\vect{x}}\\
&\\
&= \underbrace{diag(\vect{1}h)}_{\tiny{n \times n}} \underbrace{diag(\vect{f})}_{\tiny{n \times n}}\\
&\\
&= \underbrace{diag(\vect{f}h)}_{\tiny{n \times n}}\\
&\\
&= \underbrace{diag(\vect{s}(\vect{x}))}_{\tiny{n \times n}}\\
\end{align*}
}
\end{frame}


\begin{frame}{Novas operações: softmax}

\begin{align*}
\Jacobian{\vect{t}}{h}\Jacobian{h}{\vect{x}} & = \Jacobian{\vect{t}}{h}\Jacobian{h}{g}\Jacobian{g}{\vect{f}}\Jacobian{\vect{f}}{\vect{x}}\\
&\\
&= (\underbrace{\vect{f}}_{\tiny{n \times 1}} \underbrace{-\frac{1}{g^{2}}}_{\tiny{1 \times 1}}) \underbrace{\vect{1}^{\top}}_{\tiny{1 \times n}} \underbrace{diag(\vect{f})}_{\tiny{n \times n}}\\
&\\
&= \underbrace{(\vect{f} -\frac{1}{g^{2}})}_{\tiny{n \times 1}} \underbrace{\vect{f}^{\top}}_{\tiny{1 \times n}} \\
&\\
&= \begin{bmatrix} -\vect{s}_{1}^{2} & -\vect{s}_1\vect{s}_2 & \dots & -\vect{s}_1\vect{s}_n\\\vdots & \vdots & \ddots & \vdots\\ -\vect{s}_1\vect{s}_n & -\vect{s}_n\vect{s}_2 & \dots & -\vect{s}_{n}^{2}\end{bmatrix}_{\tiny{n \times n}}\\
\end{align*}

\end{frame}


\begin{frame}{Novas operações: softmax}
\large{
\begin{align*}
\Jacobian{\vect{u}}{\vect{x}} &= diag(\vect{s}(\vect{x}))  + \Jacobian{\vect{t}}{h}\Jacobian{h}{\vect{x}}\\
&\\
&= \begin{bmatrix} \vect{s}_{1}(1 - \vect{s}_{1}) & -\vect{s}_1\vect{s}_2 & \dots & -\vect{s}_1\vect{s}_n\\
-\vect{s}_2\vect{s}_1 & \vect{s}_{2}(1 - \vect{s}_{2}) & \dots & -\vect{s}_2\vect{s}_n\\
\vdots & \vdots & \ddots & \vdots\\ 
-\vect{s}_n\vect{s}_1 & -\vect{s}_n\vect{s}_2 & \dots & \vect{s}_{n}(1 - \vect{s}_{n})\end{bmatrix}_{\tiny{n \times n}}\\
\end{align*}
}
\end{frame}


\begin{frame}{Novas operações: softmax com entropia cruzada}

$\vect{x}, \vect{y} \in \mathbb{R}^{n}$

\vspace{0.1 cm}

\begin{equation*}
u = L_{SCE}(\vect{x}, \vect{y})
\end{equation*}

$\vect{\hat{y}}(\vect{x}) = \vect{s}(\vect{x})$

$\vect{f}(\vect{\hat{y}}) = log(\vect{\hat{y}})$

$\vect{g}(\vect{f}) = \vect{y} \odot \vect{f}$

$h(\vect{g}) = sum(\vect{g})$

$t(h) = -h$


$u = L_{SCE}(\vect{x}, \vect{y}) = t$

\vspace{0.1 cm}
\begin{align*}
\bullet \; \; \Jacobian{u}{\vect{x}} &= \Jacobian{t}{h} \Jacobian{h}{\vect{g}}\Jacobian{\vect{g}}{\vect{f}}\Jacobian{\vect{f}}{\vect{\hat{y}}}\Jacobian{\vect{\hat{y}}}{\vect{x}}\\
\end{align*}

\end{frame}

\begin{frame}{Novas operações: softmax com entropia cruzada}
\begin{align*}
\Jacobian{u}{\vect{x}} &= \Jacobian{t}{h} \Jacobian{h}{\vect{g}}\Jacobian{\vect{g}}{\vect{f}}\Jacobian{\vect{f}}{\vect{\hat{y}}}\Jacobian{\vect{\hat{y}}}{\vect{x}}\\
&\\
&= \underbrace{-1}_{\tiny{1 \times 1}} \underbrace{\vect{1}^{\top}}_{\tiny{1 \times n}} \underbrace{diag(\vect{y})}_{\tiny{n \times n}} \underbrace{diag(\vect{\hat{y}}^{-1})}_{\tiny{n \times n}} \underbrace{\Jacobian{\vect{\hat{y}}}{\vect{x}}}_{\tiny{n \times n}}\\
&\\
&= \underbrace{-\vect{1}^{\top}}_{\tiny{1 \times n}} \underbrace{diag(\frac{\vect{y}}{\vect{\hat{y}}})}_{\tiny{n \times n}}\underbrace{\Jacobian{\vect{\hat{y}}}{\vect{x}}}_{\tiny{n \times n}}\\
&\\
&=  \underbrace{-\frac{\vect{y}}{\vect{\hat{y}}}^{\top}}_{\tiny{1 \times n}} \underbrace{\Jacobian{\vect{\hat{y}}}{\vect{x}}}_{\tiny{n \times n}}\\
\end{align*}

\end{frame}

\begin{frame}{Novas operações: softmax com entropia cruzada}

\begin{align*}
\Jacobian{u}{\vect{x}_{i}} &= -\frac{\vect{y}}{\vect{\hat{y}}}^{\top} \Jacobian{\vect{\hat{y}}}{\vect{x}}_{:, i}\\
&\\
&= \frac{y_{1}}{\hat{y}_{1}}\hat{y}_{i}\hat{y}_{1} + \frac{y_{2}}{\hat{y}_{2}}\hat{y}_{i}\hat{y}_{2} + \dots - \frac{y_{i}}{\hat{y}_{i}}\hat{y}_{i}(1 - \hat{y}_{i}) + \dots +  \frac{y_{n}}{\hat{y}_{n}}\hat{y}_{i}\hat{y}_{n} \\
&\\
&= y_{1}\hat{y}_{i} + y_{2}\hat{y}_{i} + \dots + (y_{i}\hat{y}_{i} - y_{i}) + \dots +  y_{n}\hat{y}_{i} \\
&\\
&=  \sum_{j=1}^{n} y_{j}\hat{y}_{i} - y_{i}\\
&=  \hat{y}_{i} - y_{i} \text{ (quando } sum(\vect{y})=1 \text{)}
\end{align*}

Assim, daqui em diante $\Jacobian{u}{\vect{x}} = (\vect{\hat{y}} - \vect{y})^{\top}$
\end{frame}

\begin{frame}{Exemplo 2: Regressão logística}
\large{

$\vect{x}\in \mathbb{R}^{n}$, $\vect{W} \in \mathbb{R}^{d\times n}$ e $\vect{b}, \vect{y} \in \mathbb{R}^{d}$

\vspace{0.1 cm}

\begin{equation*}
L = L_{SCE}(\vect{W}\vect{x} + \vect{b}, \vect{y})
\end{equation*}

\vspace{0.1 cm}
$\vect{u} = \vect{W}\vect{x} + \vect{b}$

\vspace{0.1 cm}
$\vect{\hat{y}} = \vect{s}(\vect{u})$

\vspace{0.5 cm}

 $\bullet \; \; \Jacobian{L}{\vect{b}} = \Jacobian{L_{SCE}}{\vect{u}}\Jacobian{\vect{u}}{\vect{b}} = \underbrace{(\vect{\hat{y}} - \vect{y})^{\top}}_{\tiny{1 \times d}} \underbrace{\vect{I}}_{\tiny{d \times d}} = \underbrace{(\vect{\hat{y}} - \vect{y})^{\top}}_{\tiny{1 \times d}}$
}
\end{frame}

\begin{frame}{Exemplo 2: Regressão logística}
\large{

\begin{align*}
\bullet \; \; \Jacobian{L}{\vect{W}}&= \Jacobian{L_{SCE}}{\vect{u}}\Jacobian{\vect{u}}{\vect{W}}\\
&\\
&= \underbrace{(\vect{\hat{y}} - \vect{y})^{\top}}_{\tiny{1 \times d}}\underbrace{\Jacobian{\vect{u}}{\vect{W}}}_{\tiny{d \times d \times n}}\\
&\\
&= \begin{bmatrix} (\hat{y}_1 - y_1)x_1 & (\hat{y}_1 - y_1)x_2 & \dots & (\hat{y}_1 - y_1)x_n\\
(\hat{y}_2 - y_2)x_1 & (\hat{y}_2 - y_2)x_2 & \dots & (\hat{y}_2 - y_2)x_n\\
\vdots & \vdots & \ddots & \vdots\\ 
(\hat{y}_d - y_d)x_1 & (\hat{y}_d - y_d)x_2 & \dots & (\hat{y}_d - y_d)x_n\\\end{bmatrix}_{\tiny{1 \times d \times n}}\\
\end{align*}
}
\end{frame}


\begin{frame}{Exemplo 2: Regressão logística}
\Large{

\begin{align*}
\bullet \; \; \Jacobian{L}{\vect{x}}&= \Jacobian{L_{SCE}}{\vect{u}}\Jacobian{\vect{u}}{\vect{x}}\\
&\\
&= \underbrace{(\vect{\hat{y}} - \vect{y})^{\top}}_{\tiny{1 \times d}}\underbrace{\vect{W}}_{\tiny{d \times n}}\\
&\\
&= \begin{bmatrix} \sum_{j=1}^{d} (\hat{y}_j - y_j)w_{j,1}, & \dots , & \sum_{j=1}^{d} (\hat{y}_j - y_j)w_{j,1}\\\end{bmatrix}_{\tiny{1 \times n}}\\
\end{align*}
}
\end{frame}




\begin{frame}{Exemplo 3: Rede neural com uma camada escondida}
\large{

$\vect{x}\in \mathbb{R}^{n}$, $\vect{W}^{(1)} \in \mathbb{R}^{k\times n}$ e $\vect{b}^{(1)} \in \mathbb{R}^{k}$, $\vect{W}^{(2)} \in \mathbb{R}^{d\times k}$  $\vect{b}^{(2)} \in \mathbb{R}^{d}$,  $\vect{y} \in \mathbb{R}^{d}$

\vspace{0.1 cm}

\begin{equation*}
\vect{h}^{(1)} = g(\vect{W}^{(1)}\vect{x} + \vect{b}^{(1)})
\end{equation*}

\begin{equation*}
\vect{h}^{(2)} = \vect{W}^{(2)}\vect{h}^{(1)} + \vect{b}^{(2)}
\end{equation*}

\begin{equation*}
\vect{\hat{y}} = \vect{s}(\vect{h}^{(2)})
\end{equation*}

\begin{equation*}
L = L_{SCE}(\vect{h}^{(2)}, \vect{y})
\end{equation*}
}
\end{frame}

\begin{frame}{Exemplo 3: Rede neural com uma camada escondida}
\large{


\begin{equation*}
\vect{\delta} \leftarrow \Jacobian{L_{SCE}}{\vect{h}^{(2)}}
\end{equation*}

\vspace{0.5 cm}
 $\bullet \; \; \Jacobian{L}{\vect{b}^{(2)}} = \Jacobian{L_{SCE}}{\vect{h}^{(2)}}\Jacobian{\vect{h}^{(2)}}{\vect{b}^{(2)}} = \vect{\delta} \Jacobian{\vect{h}^{(2)}}{\vect{b}^{(2)}} =  \underbrace{(\vect{\hat{y}} - \vect{y})}^{\top}_{\tiny{1 \times d}} \underbrace{\vect{I}}_{\tiny{d \times d}} = \underbrace{(\vect{\hat{y}} - \vect{y})^{\top}}_{\tiny{1 \times d}}$
}
\end{frame}


\begin{frame}{Exemplo 3: Rede neural com uma camada escondida}
\large{

\begin{align*}
\bullet \; \; \Jacobian{L}{\vect{W}^{(2)}}&= \Jacobian{L_{SCE}}{\vect{h}^{(2)}}\Jacobian{\vect{h}^{(2)}}{\vect{W}^{(2)}}\\
&\\
&= \underbrace{\vect{\delta}}_{\tiny{1 \times d}}\underbrace{\Jacobian{\vect{h}^{(2)}}{\vect{W}^{(2)}}}_{\tiny{d \times d \times k}}\\
&\\
&= \begin{bmatrix} (\hat{y}_1 - y_1)h^{1}_1 & (\hat{y}_1 - y_1)h^{1}_2 & \dots & (\hat{y}_1 - y_1)h^{1}_k\\
(\hat{y}_2 - y_2)h^{1}_1 & (\hat{y}_2 - y_2)h^{1}_2 & \dots & (\hat{y}_2 - y_2)h^{1}_k\\
\vdots & \vdots & \ddots & \vdots\\ 
(\hat{y}_d - y_d)h^{1}_1 & (\hat{y}_d - y_d)h^{1}_2 & \dots & (\hat{y}_d - y_d)h^{1}_k\\\end{bmatrix}_{\tiny{1 \times d \times k}}\\
\end{align*}
}
\end{frame}

\begin{frame}{Exemplo 3: Rede neural com uma camada escondida}
\Large{

\begin{align*}
\vect{\delta}&\leftarrow \Jacobian{L_{SCE}}{\vect{h}^{(2)}}\Jacobian{\vect{h}^{(2)}}{\vect{h}^{(1)}}\\
&\\
&= \vect{\delta}\Jacobian{\vect{h}^{(2)}}{\vect{h}^{(1)}}\\
&\\
&= (\vect{\hat{y}} - \vect{y})^{\top} \vect{W}^{(2)}\\
\end{align*}
}
\end{frame}

\begin{frame}{Exemplo 3: Rede neural com uma camada escondida}
\Large{

\begin{align*}
\bullet \; \; \Jacobian{L}{\vect{b}^{(1)}}&= \Jacobian{L_{SCE}}{\vect{h}^{(2)}}\Jacobian{\vect{h}^{(2)}}{\vect{h}^{(1)}}\Jacobian{\vect{h}^{(1)}}{\vect{b}^{(1)}}\\
&\\
&= \vect{\delta}\Jacobian{\vect{h}^{(1)}}{\vect{b}^{(1)}}\\
&\\
&= \underbrace{(\vect{\hat{y}} - \vect{y})^{\top}}_{\tiny{1 \times d}} \underbrace{\vect{W}^{(2)}}_{\tiny{d \times k}} \underbrace{diag(g^{\prime}(\vect{W}^{(1)}\vect{x} + \vect{b}^{(1)}))}_{\tiny{k \times k}}\\
\end{align*}
}
\end{frame}

\begin{frame}{Exemplo 3: Rede neural com uma camada escondida}
\Large{

\begin{align*}
\bullet \; \; \Jacobian{L}{\vect{W}^{(1)}}&= \Jacobian{L_{SCE}}{\vect{h}^{(2)}}\Jacobian{\vect{h}^{(2)}}{\vect{h}^{(1)}}\Jacobian{\vect{h}^{(1)}}{\vect{W}^{(1)}}\\
&\\
&= \vect{\delta}\Jacobian{\vect{h}^{(1)}}{\vect{W}^{(1)}}\\
&\\
&= \underbrace{(\vect{\hat{y}} - \vect{y})^{\top}}_{\tiny{1 \times d}} \underbrace{\vect{W}^{(2)}}_{\tiny{d \times k}} \underbrace{\Jacobian{\vect{h}^{(1)}}{\vect{W}^{(1)}}}_{\tiny{k \times k \times n}}\\
\end{align*}
}
\end{frame}

\begin{frame}{Algoritmo de back-propagation (para redes neurais)}
\begin{algorithm}[H]
\begin{algorithmic}[1]
\STATE \textbf{Require:} $K$, network depth
\STATE \textbf{Require:} $\vect{x}$, the input to process
\STATE \textbf{Require:} $\vect{y}$, the target output
\STATE \textbf{Require:} $L( \cdot ,  \cdot)$, output with cost function
\STATE \textbf{Require:} $\vect{h}^{(i)}$, activation function for the i-th layer with weights $\vect{W}^{(i)}$ and bias $\vect{b}^{(i)}$
\STATE $\vect{\delta} \leftarrow \Jacobian{out}{\vect{h}^{(K)}}$
\FOR{$i=K$ down to $1$}
\STATE $\Jacobian{L}{\vect{b}^{(i)}} \leftarrow  \vect{\delta}\Jacobian{\vect{h}^{(i)}}{\vect{b}^{(i)}}$
\STATE $\Jacobian{L}{\vect{W}^{(i)}} \leftarrow  \vect{\delta}\Jacobian{\vect{h}^{(i)}}{\vect{W}^{(i)}}$
\STATE $\vect{\delta} \leftarrow \vect{\delta}\Jacobian{\vect{h}^{(i)}}{\vect{h}^{(i-1)}}$
\ENDFOR
\end{algorithmic}
\caption{Back-propagation for a deep neural network}
\label{alg:seq}
\end{algorithm}
\end{frame}


\begin{frame}[allowframebreaks]{Referências}

  \bibliography{my_references}
  \bibliographystyle{abbrv}

\end{frame}




\end{document}